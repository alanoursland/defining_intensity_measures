\section{What This Chapter Will Do}

The previous section highlighted a fundamental issue in many classification and measurement systems: while \textbf{distance} is a rigorously defined mathematical concept, \textbf{intensity} lacks a formal foundation. This lack of definition has led to inconsistencies in fields ranging from machine learning to cognitive science, particularly in how classification confidence and feature representations are interpreted.

This chapter aims to resolve these issues by establishing a well-defined framework for intensity based on \textbf{competitive distance}. Specifically, we will:

\begin{enumerate}
    \item \textbf{Define intensity as a function of contrastive distance.} We will show that intensity should not be treated as an absolute measure but rather as a transformation of relative distances between competing classes.
    \item \textbf{Reinterpret classification confidence.} By analyzing how classification systems operate, we will demonstrate that confidence is best understood as a function of how much closer an input is to one class than to others.
    \item \textbf{Provide a rigorous mathematical foundation.} We will formally establish how competitive distance satisfies necessary mathematical properties and aligns with well-defined concepts in metric spaces.
    \item \textbf{Explain why softmax works.} By reinterpreting softmax as a competitive distance transformation, we will clarify why neural networks behave as they do and how confidence scores emerge.
    \item \textbf{Propose improvements for machine learning models.} We will explore how explicitly incorporating competitive distance into neural networks can enhance robustness, calibration, and adversarial resistance.
    \item \textbf{Bridge artificial intelligence with cognitive science.} We will show that the competitive distance framework aligns with human perception, decision-making, and category learning, suggesting that it may be a fundamental principle of intelligence.
\end{enumerate}

\subsection{Chapter Structure}

To achieve these goals, this chapter is structured as follows:

\begin{itemize}
    \item \textbf{Section 2: The Foundation of Distance Measures} \\
    We establish the mathematical properties of distance measures and explain why distance is the fundamental quantity underlying similarity and classification.
    \item \textbf{Section 3: The Problem with Intensity} \\
    We demonstrate why intensity cannot be an independent measure, showing that absolute activation values and naive intensity interpretations lead to inconsistencies.
    \item \textbf{Section 4: Competitive Distance and Classification} \\
    We introduce a formal definition of intensity based on contrastive distance, derive decision boundaries, and explain classification confidence as a function of competitive separation.
    \item \textbf{Section 5: Implications and Applications} \\
    We explore how this framework improves neural network training, enhances robustness to adversarial attacks, and aligns with biological cognition.
    \item \textbf{Section 6: Conclusion and Future Research} \\
    We summarize key insights and propose future research directions in machine learning, neuroscience, and theoretical measurement.
\end{itemize}

\subsection{A Paradigm Shift: From Absolute Activation to Competitive Distance}

This chapter challenges the traditional view that classification systems operate based on absolute activation values or intensity measures. Instead, we propose a paradigm shift: \textbf{classification is a competitive process, and intensity should be defined as a function of relative distances between class representations.}

By the end of this chapter, we will have established a new theoretical foundation for classification confidence, one that is mathematically rigorous, biologically plausible, and practically useful for improving artificial intelligence and cognitive modeling.
