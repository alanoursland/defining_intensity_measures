\subsection{Distance vs. Intensity: The Problem}

In many fields, the concepts of \textit{distance} and \textit{intensity} are used interchangeably or without formal distinction. While distance is rigorously defined in mathematics, intensity is often treated as an intuitive, heuristic measure of strength or presence. This lack of formalism creates inconsistencies, particularly in classification systems where confidence and decision-making rely on well-defined measures.

To address these inconsistencies, we introduce the concept of \textbf{competitive distance} early in this section. Competitive distance provides a structured approach to defining classification confidence by comparing relative distances rather than relying on heuristic intensity values.

\subsubsection{Distance: A Well-Defined Measure}

Distance is one of the most fundamental concepts in mathematics. A properly defined distance function (or metric) satisfies key properties such as non-negativity, symmetry, and the triangle inequality. Some common distance measures include:

\begin{itemize}
    \item \textbf{Euclidean Distance:} Measures the straight-line separation between two points:
    \[
    d(x, y) = \sqrt{(x_1 - y_1)^2 + (x_2 - y_2)^2 + \dots + (x_n - y_n)^2}.
    \]
    \item \textbf{Mahalanobis Distance:} Adjusts for correlations and feature variances in multidimensional spaces:
    \[
    D_M(x, y) = \sqrt{(x - y)^T \Sigma^{-1} (x - y)}.
    \]
    \item \textbf{Cosine Distance:} Measures angular separation between vectors rather than absolute magnitude:
    \[
    d_{\text{cosine}}(x, y) = 1 - \frac{x \cdot y}{\|x\| \|y\|}.
    \]
\end{itemize}

These measures provide a \textbf{geometric basis for similarity and dissimilarity} in various mathematical spaces.

% Placeholder: Insert a simple diagram illustrating Euclidean vs. Mahalanobis vs. Cosine distance.

\subsubsection{Competitive Distance: A More Rigorous Alternative to Intensity}

Traditional intensity-based measures suffer from several shortcomings, particularly in classification and decision-making systems. Instead of using heuristic intensity measures, we define \textbf{competitive distance}, which quantifies classification confidence in terms of the relative separation between competing class distances.

Competitive distance is defined as:
\[
    S_c(x) = f(D_{\neg c}(x) - D_c(x)),
\]
where:
\begin{itemize}
    \item \( D_c(x) \) is the distance from \( x \) to the target class prototype.
    \item \( D_{\neg c}(x) \) is the distance from \( x \) to the nearest non-target class prototype.
    \item \( S_c(x) \) represents a properly defined classification confidence measure based on relative distances.
\end{itemize}

This formulation ensures that classification confidence:
\begin{itemize}
    \item Is grounded in geometric relationships between classes.
    \item Remains bounded and interpretable.
    \item Naturally aligns with decision boundaries and uncertainty estimation.
\end{itemize}

% Placeholder: Add a visualization showing decision boundaries emerging from competitive distance comparisons.

\subsubsection{Why Intensity is a Concept Without Formal Definition}

Unlike distance, intensity is often used informally to describe the strength of a feature, probability, or classification confidence. Examples include:

\begin{itemize}
    \item \textbf{Brightness in Physics:} The intensity of light is often equated with amplitude squared.
    \item \textbf{Neural Network Activations:} The magnitude of an activation is assumed to correspond to the presence of a feature.
    \item \textbf{Classification Confidence:} Softmax outputs are interpreted as confidence scores, but their underlying mathematical meaning is ambiguous.
\end{itemize}

Unlike distance, intensity lacks a universal definition, leading to ambiguity in how it should be interpreted across different domains.

% Placeholder: Add a table comparing distance-based and intensity-based measures across different disciplines.

\subsubsection{Why the Traditional View of Intensity is Problematic}

Interpreting intensity as an independent measure leads to several problems:

\begin{enumerate}
    \item \textbf{Lack of Scale Consistency:} Intensity is often defined in arbitrary units, making comparisons across different models or datasets difficult.
    \item \textbf{Unbounded Growth:} If intensity were simply a negation of distance, then maximizing intensity would drive distances toward negative infinity, which is not meaningful.
    \item \textbf{Overconfidence in Classification:} Traditional activation-based confidence measures can exaggerate certainty due to improper scaling.
    \item \textbf{Lack of Decision Boundary Awareness:} Intensity-based confidence does not inherently reflect how close a sample is to a decision boundary.
\end{enumerate}

% Placeholder: Add a small example illustrating how neural activations fail to represent decision boundary confidence.

\subsubsection{The Need for a Competitive Distance Framework}

Since intensity lacks a foundational definition, we propose redefining it in terms of \textit{contrastive distance}. Instead of treating intensity as an absolute measure, we introduce a framework where classification confidence is derived from the \textbf{difference between distances to competing classes}:

\[
S_c(x) = f(D_{\neg c}(x) - D_c(x)).
\]

This formulation ensures that classification confidence is not an arbitrary activation magnitude but a function of how much closer an input is to one class compared to others. It naturally prevents unbounded confidence values, aligns with decision boundaries, and improves uncertainty estimation.

% Placeholder: Add a simple visualization of competitive distance illustrating how decision boundaries emerge from relative distances.

\subsubsection{Conclusion and Transition}

The lack of a formal definition for intensity has led to inconsistencies in classification confidence, neural activations, and probability scaling. By reframing intensity in terms of \textbf{competitive distance}, we provide a rigorous and interpretable measure that resolves these issues. The next section will outline the goals of this chapter and the key insights that will be developed in later sections.
